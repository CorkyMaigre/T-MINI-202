
\documentclass[]{article}

\usepackage[T1]{fontenc}
\usepackage{lmodern}
\usepackage{amssymb,amsmath}
\usepackage{ifxetex,ifluatex}
\usepackage{fixltx2e} % provides \textsubscript
% \usepackage[top=2.5cm, bottom=2.5cm, left=2cm, right=2cm]{geometry}
\usepackage{graphics,graphicx} % Required for including pictures
\usepackage{pgfplots}
\usepackage{tikz}
\usepackage{xcolor} % Required for specifying colors by name
\usepackage{verbatim}

\makeatletter
\newcommand*{\toccontents}{\@starttoc{toc}}
\makeatother

% use upquote if available, for straight quotes in verbatim environments
\IfFileExists{upquote.sty}{\usepackage{upquote}}{}
\ifnum 0\ifxetex 1\fi\ifluatex 1\fi=0 % if pdftex
  \usepackage[utf8]{inputenc}
\else % if luatex or xelatex
  \ifxetex
    \usepackage{mathspec}
    \usepackage{xltxtra,xunicode}
  \else
    \usepackage{fontspec}
  \fi
  \defaultfontfeatures{Mapping=tex-text,Scale=MatchLowercase}
  \newcommand{\euro}{€}
\fi
% use microtype if available
\IfFileExists{microtype.sty}{\usepackage{microtype}}{}
\usepackage{longtable,booktabs}
\ifxetex
  \usepackage[setpagesize=false, % page size defined by xetex
              unicode=false, % unicode breaks when used with xetex
              xetex]{hyperref}
\else
  \usepackage[unicode=true]{hyperref}
\fi
\hypersetup{breaklinks=true,
            bookmarks=true,
            pdfauthor={},
            pdftitle={},
            colorlinks=true,
            citecolor=blue,
            urlcolor=blue,
            linkcolor=magenta,
            pdfborder={0 0 0}}
\urlstyle{same}  % don't use monospace font for urls
\setlength{\parindent}{0pt}
\setlength{\parskip}{6pt plus 2pt minus 1pt}
\setlength{\emergencystretch}{3em}  % prevent overfull lines
\setcounter{secnumdepth}{0}

% \myFigure
% {Title}{url}{scale}
\newcommand\myFigure[3]{%
	\begin{figure}[h]
		\begin{center}
			\includegraphics[scale=#3]{img/#2}
		\end{center}
		\setlength\abovecaptionskip{-1px}
		\caption{#1.\label{fig:#2}}
	\end{figure}
}


\title{Multithreading Benchmarks}
\author{Corky Maigre}
\date{\today}


\begin{document}

\maketitle

% \begin{abstract}
% sfsdfsd
% \end{abstract}

\toccontents





\section{Introduction}\label{introduction}

This project intended to compare the performance of a C/C++ console
application using CPU \& GPU programming. The console application
consisted of doing a square matrix multiplication via several methods :
\begin{itemize}
	\item CPU Programming
	\begin{itemize}
		\item Sequential code
		\item Multithreading code (one thread per cell of the result matrix)
		\item Multithreading code (one thread per row of the result matrix)
		\item Multithreading code (each cell of the result matrix is assigned to one of the \emph{n} threads)
	\end{itemize}
	\item GPU Programming
\end{itemize}


Each programming method was compared each other by doing a benchmark on
the elapsed time for the computation only.

Information : - \href{https://github.com/CorkyMaigre}{Homepage} -
\href{https://github.com/CorkyMaigre/multithreading-benchmarks}{GitHub} - \href{http://www.corkymaigre.be/}{Website}


\hyperdef{}{configuration}{\section{Configuration}\label{configuration}}

\hyperdef{}{hardware-specifications}{\subsection{Hardware
Specifications}\label{hardware-specifications}}

Using an ASUS X93S Series laptop whose the charasteristics are listed
below :

\begin{longtable}[c]{@{}ll@{}}
\toprule\addlinespace
Charasteristics & Description
\\\addlinespace
\midrule\endhead
Processor & Intel Core i5 2430M 2.4 GHz \textasciitilde{} 2.9 GHz
\\\addlinespace
Operating System & Windows 7 Home Premium
\\\addlinespace
Chipset & Intel HM65 Express
\\\addlinespace
Memory & DDR3 1333 MHz SDRAM, 4096 MB, (1 x 4096 MB)
\\\addlinespace
Display & 18.4" 16:9 Full HD (1920x1080) LED Backlight
\\\addlinespace
Graphic & NVIDIA® GeForce® GT 540M with 1GB DDR3 VRAM
\\\addlinespace
Storage & 1 TB 7200 rpm
\\\addlinespace
Optical Drive & DVD player
\\\addlinespace
Card Reader & Card reader ( SD/ SDHC/ MS/ MS Pro/ MMC)
\\\addlinespace
Webcam & 0.3 Mega Pixel Fixed web camera
\\\addlinespace
Networking & Integrated 802.11 b/g/n, Bluetooth™ V2.1+EDR,
\\\addlinespace
&10/100/1000 Base T
\\\addlinespace
Interface & 1 x Microphone-in jack, 1 x Headphone-out jack,
\\\addlinespace
&
1 x VGA port
/ Mini D-sub 15 pins for external monitor,
\\\addlinespace
&  1 x USB 3.0 port, 3 x USB 2.0
ports, 1 x RJ45 LAN Jack
\\\addlinespace
& for LAN insert, 1 x HDMI
\\\addlinespace
Audio & Built-in Speakers And Microphone, SonicFocus,
\\\addlinespace
& Altec Lansing® Speakers
\\\addlinespace
Battery & 6Cells : 5200 mAh 56 Whrs
\\\addlinespace
Power Adapter & Output : 19 V DC, 6.3 A, 120 W Input : 100 -240 V AC,
\\\addlinespace
& 50/60 Hz universal
\\\addlinespace
Dimensions & 44.1 x 29.5 x 4.23 \textasciitilde{}5.59 cm (WxDxH)
\\\addlinespace
Weight & 4.11 kg (with 6 cell battery)
\\\addlinespace
Note & Master HDD: 3.5'' SATA, Second HDD: 2.5'' SATA
\\\addlinespace
\bottomrule
\\\addlinespace
\caption{Hardware specifications.}
\end{longtable}

\hyperdef{}{ide-configuration}{\subsection{IDE
Configuration}\label{ide-configuration}}

For this project, Visual Studio Community 2015 was used for CPU
programming with the pthread library and Visual Studio 2013 was used for
GPU programming with CUDA since CUDA is not supported in VC 2015.

\begin{quote}
\textbf{CAUTION} Visual Studio Community 2015 does not support CUDA yet.
You need to use an older version of VC such as Visual Studio 2012 \& 2013.
\end{quote}

First of all, you have to set that pthread library is used by typing
`pthreadVC2.lib' into the additional dependencies found at \verb/Property/
\textgreater{} \verb/Links Editor/ \textgreater{} \verb/Additional Dependencies/
as shown on the picture \ref{fig:config-000}.

\myFigure{Using pthread}{config-000}{0.4}

\begin{quote}
\textbf{CAUTION} It is possible that you need to type
\verb/HAVE STRUCT TIMESPEC/ into the preprocessor definition found at
\verb/Property/ \textgreater{} C/C++ \textgreater{} \verb/Preprocessor/
\textgreater{} \verb/Preprocessor Definition/ as shown on the picture \ref{fig:config-001}.
Otherwise you will have this error message: Error C2011 `timespec'~:
redefinition of type `struct'
\end{quote}

\myFigure{Timespec error}{config-001}{0.4}


\hyperdef{}{benchmarks}{\section{Benchmarks}\label{benchmarks}}

All benchmarks presented here are resulted from a specific use of
threading such as the number of threads doing the computation.

\hyperdef{}{benchmark-1}{\subsection{Benchmark 1}\label{benchmark-1}}

This first benchmark is created by executing the C++ sequential code
with CPU. The code uses dynamic arrays and a `Matrix' structure.
The elapsed time is determined by an average of three tests.

\begin{longtable}[c]{@{}ccc@{}}
\caption{Execution times for benchmark 1.}
\\\addlinespace
\toprule\addlinespace
Matrix Dimension & Number of cells & Average Elapsed Time 
\\\addlinespace
\midrule\endhead
2 x 2 & 4 & 0.000000 s 
\\\addlinespace
25 x 25 & 625 & 0.000149 s 
\\\addlinespace
50 x 50 & 2,500 & 0.001165 s 
\\\addlinespace
100 x 100 & 10,000 & 0.008968 s 
\\\addlinespace
200 x 200 & 40,000 & 0.071950 s 
\\\addlinespace
500 x 500 & 250,000 & 0.934330 s 
\\\addlinespace
1,000 x 1,000 & 1,000,000 & 15.01570 s
\\\addlinespace
1,500 x 1,500 & 2,250,000 & 55.67037 s
\\\addlinespace
2,000 x 2,000 & 4,000,000 & 142.5680 s
\\\addlinespace
\bottomrule
\end{longtable}


\hyperdef{}{benchmark-2}{\subsection{Benchmark 2}\label{benchmark-2}}

The second benchmark is created by executing the parallel code with CPU
in C++ using pthread. The code uses dynamic arrays and a dynamic number
of threads. Each cell in the result matrix is performed by one thread.
The elapsed time is determined by an average of three tests.

\begin{longtable}[c]{@{}ccc@{}}
\caption{Execution times for benchmark 2.}
\\\addlinespace
\toprule\addlinespace
Matrix Dimension & Number of cells & Average Elapsed Time
\\\addlinespace
\midrule\endhead
2 x 2 & 4 & 0.001207 s 
\\\addlinespace
25 x 25 & 625 & 0.558716 s
\\\addlinespace
50 x 50 & 2,500 & 1.841263 s
\\\addlinespace
100 x 100 & 10,000 & 8.092927 s
\\\addlinespace
200 x 200 & 40,000 & 33.41967 s
\\\addlinespace
500 x 500 & 250,000 & 213.4013 s
\\\addlinespace
1,000 x 1,000 & 1,000,000 & 936.1580 s 
\\\addlinespace
1,500 x 1,500 & 2,250,000 & 0.000000 s 
\\\addlinespace
2,000 x 2,000 & 4,000,000 & 0.000000 s
\\\addlinespace
\bottomrule
\end{longtable}

Before launching the console application for a square matrix of 1,000 x
1,000, the memory is low (but Visual Studio takes a big part of memory).

At the end of the console application for a square matrix of 1,000 x
1,000, the memory is full because all pointers take a lot of space and
there are 1,000,000 of threads in memory. When the application finished, all the memory is released.

\begin{quote}
\textbf{NOTE} The execution times for 1500x1500 and 2000x2000 matrice dimension has not been calculated since it takes more than 15 minutes.
\end{quote}


\hyperdef{}{benchmark-3}{\subsection{Benchmark 3}\label{benchmark-3}}

The third benchmark is created by executing the parallel code with CPU
in C++ using pthread. The code uses dynamic arrays and a dynamic number
of threads. Each row in the result matrix is performed by one thread.
The elapsed time is determined by an average of three tests.

\begin{longtable}[c]{@{}ccc@{}}
\caption{Execution times for benchmark 3.}
\\\addlinespace
\toprule\addlinespace
Matrix Dimension & Number of cells & Average Elapsed Time
\\\addlinespace
\midrule\endhead
2 x 2 & 4 & 0.000451 s
\\\addlinespace
25 x 25 & 625 & 0.012860 s
\\\addlinespace
50 x 50 & 2,500 & 0.025046 s
\\\addlinespace
100 x 100 & 10,000 & 0.057352 s
\\\addlinespace
200 x 200 & 40,000 & 0.112013 s
\\\addlinespace
500 x 500 & 250,000 & 1.108248 s
\\\addlinespace
1,000 x 1,000 & 1,000,000 & 8.496977 s
\\\addlinespace
1,500 x 1,500 & 2,250,000 & 26.31603 s
\\\addlinespace
2,000 x 2,000 & 4,000,000 & 44.62573 s
\\\addlinespace
\bottomrule
\end{longtable}

\newpage

\hyperdef{}{benchmark-4}{\subsection{Benchmark 4}\label{benchmark-4}}

The fourth benchmark is created by executing the parallel code with CPU
in C++ using pthread. The code uses dynamic arrays and a dynamic number
of threads. Cells in the result matrix are performed by one of the
threads according to a shifting algorithm.

\subsubsection{Number of threads: 2}\label{number-of-threads-2}
The elapsed time is determined by an average of three tests.

\begin{longtable}[c]{@{}ccc@{}}
\caption{Execution times for benchmark 4 with 2 threads.}
\\\addlinespace
\toprule\addlinespace
Matrix Dimension & Number of cells & Average Elapsed Time
\\\addlinespace
\midrule\endhead
2 x 2 & 4 & 0.002106 s
\\\addlinespace
25 x 25 & 625 & 0.001467 s
\\\addlinespace
50 x 50 & 2,500 & 0.003618 s
\\\addlinespace
100 x 100 & 10,000 & 0.023646 s
\\\addlinespace
200 x 200 & 40,000 & 0.051373 s
\\\addlinespace
500 x 500 & 250,000 & 1.031829 s
\\\addlinespace
1,000 x 1,000 & 1,000,000 & 10.90120 s
\\\addlinespace
1,500 x 1,500 & 2,250,000 & 38.62227 s
\\\addlinespace
2,000 x 2,000 & 4,000,000 & 101.9047 s
\\\addlinespace
\bottomrule
\end{longtable}

\subsubsection{Number of threads: 3}\label{number-of-threads-3}
The elapsed time is determined by an average of three tests.

\begin{longtable}[c]{@{}ccc@{}}
\caption{Execution times for benchmark 4 with 3 threads.}
\\\addlinespace
\toprule\addlinespace
Matrix Dimension & Number of cells & Average Elapsed Time
\\\addlinespace
\midrule\endhead
2 x 2 & 4 & 0.011413 s
\\\addlinespace
25 x 25 & 625 & 0.003701 s
\\\addlinespace
50 x 50 & 2,500 & 0.0042903 s
\\\addlinespace
100 x 100 & 10,000 & 0.009855 s
\\\addlinespace
200 x 200 & 40,000 & 0.051281 s
\\\addlinespace
500 x 500 & 250,000 & 0.733569 s
\\\addlinespace
1,000 x 1,000 & 1,000,000 & 7.827880 s
\\\addlinespace
1,500 x 1,500 & 2,250,000 & 30.25893 s
\\\addlinespace
2,000 x 2,000 & 4,000,000 & 87.41533 s
\\\addlinespace
\bottomrule
\end{longtable}

\subsubsection{Number of threads: 4}\label{number-of-threads-4}
The elapsed time is determined by an average of three tests.

\begin{longtable}[c]{@{}ccc@{}}
\caption{Execution times for benchmark 4 with 4 threads.}
\\\addlinespace
\toprule\addlinespace
Matrix Dimension & Number of cells & Average Elapsed Time
\\\addlinespace
\midrule\endhead
2 x 2 & 4 & 0.123637 s
\\\addlinespace
25 x 25 & 625 & 0.002267 s
\\\addlinespace
50 x 50 & 2,500 & 0.003300 s
\\\addlinespace
100 x 100 & 10,000 & 0.027753 s
\\\addlinespace
200 x 200 & 40,000 & 0.058286 s
\\\addlinespace
500 x 500 & 250,000 & 0.668145 s
\\\addlinespace
1,000 x 1,000 & 1,000,000 & 7.767977 s
\\\addlinespace
1,500 x 1,500 & 2,250,000 & 25.13680 s
\\\addlinespace
2,000 x 2,000 & 4,000,000 & 75.09370 s
\\\addlinespace
\bottomrule
\end{longtable}

\subsubsection{Number of threads: 8}\label{number-of-threads-8}

\begin{longtable}[c]{@{}ccc@{}}
\caption{Execution times for benchmark 4 with 8 threads.}
\\\addlinespace
\toprule\addlinespace
Matrix Dimension & Number of cells & Average Elapsed Time
\\\addlinespace
\midrule\endhead
2 x 2 & 4 & 0.015070 s
\\\addlinespace
25 x 25 & 625 & 0.031699 s
\\\addlinespace
50 x 50 & 2,500 & 0.006153 s
\\\addlinespace
100 x 100 & 10,000 & 0.013858 s
\\\addlinespace
200 x 200 & 40,000 & 0.048016 s
\\\addlinespace
500 x 500 & 250,000 & 0.651807 s
\\\addlinespace
1,000 x 1,000 & 1,000,000 & 8.183037 s
\\\addlinespace
1,500 x 1,500 & 2,250,000 & 30.89937 s
\\\addlinespace
2,000 x 2,000 & 4,000,000 & 90.26363 s
\\\addlinespace
\bottomrule
\end{longtable}

\begin{quote}
\textbf{NOTE} Using the `Cell' structure.
\end{quote}

\hyperdef{}{benchmark-5}{\subsection{Benchmark 5}\label{benchmark-5}}

The fifth benchmark is created by executing the parallel code with GPU
in CUDA C++. The code uses dynamic arrays and a dynamic
number of threads. Each cell in the result matrix is performed by one
thread.
The elapsed time is determined by an average of three tests.

\begin{longtable}[c]{@{}ccc@{}}
\caption{Execution times for benchmark 5.}
\\\addlinespace
\toprule\addlinespace
Matrix Dimension & Number of cells & Average Elapsed Time
\\\addlinespace
\midrule\endhead
2 x 2 & 4 & 0.000098 s
\\\addlinespace
25 x 25 & 625 & 0.000110 s
\\\addlinespace
50 x 50 & 2,500 & 0.000174 s
\\\addlinespace
100 x 100 & 10,000 & 0.000537 s
\\\addlinespace
200 x 200 & 40,000 & 0.004506 s
\\\addlinespace
500 x 500 & 250,000 & 0.066445 s
\\\addlinespace
1,000 x 1,000 & 1,000,000 & 0.537397 s
\\\addlinespace
1,500 x 1,500 & 2,250,000 & 1.851377 s
\\\addlinespace
2,000 x 2,000 & 4,000,000 & 3.164927 s
\\\addlinespace
\bottomrule
\end{longtable}




\hyperdef{}{conclusion}{\section{Comparison}\label{conclusion}}

Consider the dimensions below for all benchmarks :
\begin{itemize}
	\item 200 x 200
	\item 500 x 500
	\item 1,000 x 1,000
	\item 1,500 x 1,500
	\item 2,000 x 2,000
\end{itemize}

\begin{figure}[h]
\centering
\begin{tikzpicture}[scale=0.5]
	\foreach \x in {0, 5, ..., 20} \draw(\x,15)node[above]{\x};
	\foreach \n/\y in
	{
		Benchmark 5/13,
		Benchmark 4/10,
		Benchmark 3/7,
		Benchmark 2/4,
		Benchmark 1/1
	}
	\draw (0,\y) node[left] {\n};
	\draw (0,-1.5) grid[xstep=1,ystep=1.5] (20,15);
	\draw[line width=2mm,color=red!50,yshift=8mm] plot[xcomb] file {data/dim200-1-50.txt};
	\draw[line width=2mm,color=green!50,yshift=4mm] plot[xcomb] file {data/dim500-1-50.txt};
	\draw[line width=2mm,color=blue!50,yshift=0mm] plot[xcomb] file {data/dim1000-1-50.txt};
	\draw[line width=2mm,color=orange!50,yshift=-4mm] plot[xcomb] file {data/dim1500-1-50.txt};
	\draw[line width=2mm,color=purple!50,yshift=-8mm] plot[xcomb] file {data/dim2000-1-50.txt};
	\draw (15,11)node[fill=red!50, above]{200 x 200};
	\draw (15,10)node[fill=green!50, above]{500 x 500};
	\draw (15,9)node[fill=blue!50, above]{1000 x 1000};
	\draw (15,8)node[fill=orange!50, above]{1500 x 1500};
	\draw (15,7)node[fill=purple!50, above]{2000 x 2000};
\end{tikzpicture}
\caption{Execution time in milliseconds (scale: $\frac{1}{50}$)}
\end{figure}

\begin{figure}[h]
\centering
\begin{tikzpicture}[scale=0.5]
	\foreach \x in {0, 5, ..., 20} \draw(\x,15)node[above]{\x};
	\foreach \n/\y in
	{
		Benchmark 5/13,
		Benchmark 4/10,
		Benchmark 3/7,
		Benchmark 1/1
	}
	\draw (0,\y) node[left] {\n};
	\draw (0,-1.5) grid[xstep=1,ystep=1.5] (20,15);	
	\draw[line width=2mm,color=red!50,yshift=8mm] plot[xcomb] file {data/dim200-1-7.txt};
	\draw[line width=2mm,color=green!50,yshift=4mm] plot[xcomb] file {data/dim500-1-7.txt};
	\draw[line width=2mm,color=blue!50,yshift=0mm] plot[xcomb] file {data/dim1000-1-7.txt};
	\draw[line width=2mm,color=orange!50,yshift=-4mm] plot[xcomb] file {data/dim1500-1-7.txt};
	\draw[line width=2mm,color=purple!50,yshift=-8mm] plot[xcomb] file {data/dim2000-1-7.txt};
	\draw (15,11)node[fill=red!50, above]{200 x 200};
	\draw (15,10)node[fill=green!50, above]{500 x 500};
	\draw (15,9)node[fill=blue!50, above]{1000 x 1000};
	\draw (15,8)node[fill=orange!50, above]{1500 x 1500};
	\draw (15,7)node[fill=purple!50, above]{2000 x 2000};	
\end{tikzpicture}
\caption{Execution time in milliseconds (scale: $\frac{1}{7}$)}
\end{figure}







\hyperdef{}{conclusion}{\section{Conclusion}\label{conclusion}}


In conclusion, threads have to be well assigned for a better performance. For instance, the execution of
the code of benchmark 2 takes a long time because each cell of the result matrice is computed by an
unique thread. Thus, there are a lot of threads ($n\times n$) and the operating system must open and close them
via a preemptive method since the number of threads is limited by the machine power.
In this case, the execution of the sequential code
(\emph{i.e.}, no threads) of benchmark 1 is faster than the one of the code of benchmark 2.
~\\

Besides, when the matrices dimension are too small, the use of threads has not a big impact on the performance since
only a little gain of performance has been reported.
~\\

However, when threads are well assigned, we can massively gain in performance. For example, the benchmark 3 is better than the
two previous benchmarks and there are $n$ threads since only one thread per row is created.
~\\

We can go further with benchmark 4 by assigning a static number of threads for the matrices computation.
This allows to set the number of threads of the machine. For example, we tested 2, 3, 4, and 8 threads on benchmark 4, and the
result is that using four threads gives the best performance. It is logic because the test machines has only four threads.
The threads are alternately assigned to the result matrice cells via an custom algorithm.
~\\


Finally, GPU programming largely beats CPU programming on performance with very low execution times compared to
the ones of CPU codes.



\hyperdef{}{bugs}{\section{Bugs}\label{bugs}}

\begin{itemize}
\itemsep1pt\parskip0pt\parsep0pt
\item
  Error with the memory deleting:

  \begin{itemize}
  \itemsep1pt\parskip0pt\parsep0pt
  \item
    detected the 01 May 2016
  \item
    solved the 01 May 2016
  \end{itemize}
\item
  Error on benchmark 4 with matrix n x n where n \textgreater{} 2

  \begin{itemize}
  \itemsep1pt\parskip0pt\parsep0pt
  \item
    detected the 30 April 2016
  \item
    solved the 19 May 2016 by using pthread\_join() function.
  \end{itemize}
\end{itemize}












\end{document}
