\documentclass[]{article}
\usepackage[T1]{fontenc}
\usepackage{lmodern}
\usepackage{amssymb,amsmath}
\usepackage{ifxetex,ifluatex}
\usepackage{fixltx2e} % provides \textsubscript
% use upquote if available, for straight quotes in verbatim environments
\IfFileExists{upquote.sty}{\usepackage{upquote}}{}
\ifnum 0\ifxetex 1\fi\ifluatex 1\fi=0 % if pdftex
  \usepackage[utf8]{inputenc}
\else % if luatex or xelatex
  \ifxetex
    \usepackage{mathspec}
    \usepackage{xltxtra,xunicode}
  \else
    \usepackage{fontspec}
  \fi
  \defaultfontfeatures{Mapping=tex-text,Scale=MatchLowercase}
  \newcommand{\euro}{€}
\fi
% use microtype if available
\IfFileExists{microtype.sty}{\usepackage{microtype}}{}
\usepackage{longtable,booktabs}
\ifxetex
  \usepackage[setpagesize=false, % page size defined by xetex
              unicode=false, % unicode breaks when used with xetex
              xetex]{hyperref}
\else
  \usepackage[unicode=true]{hyperref}
\fi
\hypersetup{breaklinks=true,
            bookmarks=true,
            pdfauthor={},
            pdftitle={},
            colorlinks=true,
            citecolor=blue,
            urlcolor=blue,
            linkcolor=magenta,
            pdfborder={0 0 0}}
\urlstyle{same}  % don't use monospace font for urls
\setlength{\parindent}{0pt}
\setlength{\parskip}{6pt plus 2pt minus 1pt}
\setlength{\emergencystretch}{3em}  % prevent overfull lines
\setcounter{secnumdepth}{0}

\author{}
\date{}

\begin{document}

\section{Multithreading Benchmarks}\label{multithreading-benchmarks}

This project intended to compare the performance of a C/C++ console
application using CPU \& GPU programming. The console application
consisted of doing a square matrix multiplication via several methods :
- {[}CPU Programming{]} - {[}Sequential code{]} - {[}Multithreading code
(one thread per cell of the result matrix){]} - {[}Multithreading code
(one thread per row of the result matrix){]} - {[}Multithreading code
(each cell of the result matrix is assigned to one of the \emph{n}
threads){]} - {[}GPU Programming{]}

Each programming method was compared each other by doing a benchmark on
the elapsed time for the computation only.

Information : - \href{https://github.com/CorkyMaigre}{Homepage} -
\href{https://github.com/CorkyMaigre/multithreading-benchmarks}{Source
files} - \href{http://www.corkymaigre.be/}{Website}

\hyperdef{}{table-of-contents}{\section{Table of
contents}\label{table-of-contents}}

\begin{itemize}
\itemsep1pt\parskip0pt\parsep0pt
\item
  \hyperref[table-of-contents]{Table of contents}
\item
  \hyperref[configuration]{Configuration}

  \begin{itemize}
  \itemsep1pt\parskip0pt\parsep0pt
  \item
    \hyperref[hardware-specifications]{Hardware Specifications}
  \item
    \hyperref[ide-configuration]{IDE Configuration}
  \end{itemize}
\item
  \hyperref[benchmarks]{Benchmarks}

  \begin{itemize}
  \itemsep1pt\parskip0pt\parsep0pt
  \item
    \hyperref[benchmark-1]{Benchmark 1}
  \item
    \hyperref[benchmark-2]{Benchmark 2}
  \item
    \hyperref[benchmark-3]{Benchmark 3}
  \item
    \hyperref[benchmark-4]{Benchmark 4}
  \item
    \hyperref[benchmark-5]{Benchmark 5}
  \end{itemize}
\item
  \hyperref[conclusion]{Conclusion}
\item
  \hyperref[contribute]{Contribute}
\item
  \hyperref[bugs]{Bugs}
\end{itemize}

\hyperdef{}{configuration}{\section{Configuration}\label{configuration}}

\hyperdef{}{hardware-specifications}{\subsection{Hardware
Specifications}\label{hardware-specifications}}

Using an ASUS X93S Series laptop whose the charasteristics are listed
below :

\begin{longtable}[c]{@{}ll@{}}
\toprule\addlinespace
Charasteristics & Description
\\\addlinespace
\midrule\endhead
Processor & Intel Core i5 2430M 2.4 GHz \textasciitilde{} 2.9 GHz
\\\addlinespace
Operating System & Windows 7 Home Premium
\\\addlinespace
Chipset & Intel HM65 Express
\\\addlinespace
Memory & DDR3 1333 MHz SDRAM, 4096 MB, (1 x 4096 MB)
\\\addlinespace
Display & 18.4" 16:9 Full HD (1920x1080) LED Backlight
\\\addlinespace
Graphic & NVIDIA® GeForce® GT 540M with 1GB DDR3 VRAM
\\\addlinespace
Storage & 1 TB 7200 rpm
\\\addlinespace
Optical Drive & DVD player
\\\addlinespace
Card Reader & Card reader ( SD/ SDHC/ MS/ MS Pro/ MMC)
\\\addlinespace
Webcam & 0.3 Mega Pixel Fixed web camera
\\\addlinespace
Networking & Integrated 802.11 b/g/n, Bluetooth™ V2.1+EDR, 10/100/1000
Base T
\\\addlinespace
Interface & 1 x Microphone-in jack, 1 x Headphone-out jack, 1 x VGA port
/ Mini D-sub 15 pins for external monitor, 1 x USB 3.0 port, 3 x USB 2.0
ports, 1 x RJ45 LAN Jack for LAN insert, 1 x HDMI
\\\addlinespace
Audio & Built-in Speakers And Microphone, SonicFocus, Altec Lansing®
Speakers
\\\addlinespace
Battery & 6Cells : 5200 mAh 56 Whrs
\\\addlinespace
Power Adapter & Output : 19 V DC, 6.3 A, 120 W Input : 100 -240 V AC,
50/60 Hz universal
\\\addlinespace
Dimensions & 44.1 x 29.5 x 4.23 \textasciitilde{}5.59 cm (WxDxH)
\\\addlinespace
Weight & 4.11 kg (with 6 cell battery)
\\\addlinespace
Note & Master HDD: 3.5'' SATA, Second HDD: 2.5'' SATA
\\\addlinespace
\bottomrule
\end{longtable}

\hyperdef{}{ide-configuration}{\subsection{IDE
Configuration}\label{ide-configuration}}

For this project, Visual Studio Community 2015 was used for CPU
programming with the pthread library and Visual Studio 2013 was used for
GPU programming with CUDA since CUDA is not supported in VC 2015.

\begin{quote}
\textbf{CAUTION} Visual Studio Community 2015 does not support CUDA yet.
You need to use an older version of VC such as Visual Studio 2013.
\end{quote}

First of all, you have to set that pthread library is used by typing
`pthreadVC2.lib' into the additional dependencies found at `Property'
\textgreater{} `Links Editor' \textgreater{} `Additional Dependencies'
as shown on the picture below.

\begin{quote}
\textbf{CAUTION} It is possible that you need to type
`HAVE\_STRUCT\_TIMESPEC' into the preprocessor definition found at
`Property' \textgreater{} `C/C++' \textgreater{} `Preprocessor'
\textgreater{} `Preprocessor Definition' as shown on the picture below.
Otherwise you will have this error message: Error C2011 `timespec'~:
redefinition of type `struct'
\end{quote}

\hyperdef{}{benchmarks}{\section{Benchmarks}\label{benchmarks}}

All benchmarks presented here are resulted from a specific use of
threading such as the number of threads doing the computation.

\hyperdef{}{benchmark-1}{\subsection{Benchmark 1}\label{benchmark-1}}

This first benchmark is created by executing the C++ sequential code
with CPU. The code uses dynamic arrays and a `Matrix' structure.

\begin{longtable}[c]{@{}lllll@{}}
\toprule\addlinespace
Matrix Dimension & Number of cells & Elapsed Time 1 & Elapsed Time 2 &
Elapsed Time 3
\\\addlinespace
\midrule\endhead
2 x 2 & 4 & 0.000000 s & 0.000000 s & 0.000000 s
\\\addlinespace
25 x 25 & 625 & 0.000144 s & 0.000106 s & 0.000198 s
\\\addlinespace
50 x 50 & 2,500 & 0.001092 s & 0.001143 s & 0.001261 s
\\\addlinespace
100 x 100 & 10,000 & 0.007243 s & 0.009130 s & 0.010531 s
\\\addlinespace
200 x 200 & 40,000 & 0.069280 s & 0.050119 s & 0.096451 s
\\\addlinespace
500 x 500 & 250,000 & 0.908748 s & 0.976781 s & 0.917461 s
\\\addlinespace
1,000 x 1,000 & 1,000,000 & 14.82270 s & 14.90280 s & 15.32160 s
\\\addlinespace
1,500 x 1,500 & 2,250,000 & 51.84170 s & 53.78350 s & 61.38590 s
\\\addlinespace
2,000 x 2,000 & 4,000,000 & 144.4250 s & 144.3840 s & 138.8950 s
\\\addlinespace
\bottomrule
\end{longtable}

\hyperdef{}{benchmark-2}{\subsection{Benchmark 2}\label{benchmark-2}}

The second benchmark is created by executing the parallel code with CPU
in C++ using pthread. The code uses dynamic arrays and a dynamic number
of threads. Each cell in the result matrix is performed by one thread.

\begin{longtable}[c]{@{}lllll@{}}
\toprule\addlinespace
Matrix Dimension & Number of cells & Elapsed Time 1 & Elapsed Time 2 &
Elapsed Time 3
\\\addlinespace
\midrule\endhead
2 x 2 & 4 & 0.001818 s & 0.000263 s & 0.001539 s
\\\addlinespace
25 x 25 & 625 & 0.651238 s & 0.641366 s & 0.383544 s
\\\addlinespace
50 x 50 & 2,500 & 1.956470 s & 1.784130 s & 1.783190 s
\\\addlinespace
100 x 100 & 10,000 & 8.355190 s & 8.410570 s & 7.513020 s
\\\addlinespace
200 x 200 & 40,000 & 32.00290 s & 33.54810 s & 34.70680 s
\\\addlinespace
500 x 500 & 250,000 & 210.6790 s & 212.8600 s & 216.6650 s
\\\addlinespace
1,000 x 1,000 & 1,000,000 & 958.3180 s & 918.7720 s & 931.3840 s
\\\addlinespace
1,500 x 1,500 & 2,250,000 & 0.000000 s & 0.000000 s & 0.000000 s
\\\addlinespace
2,000 x 2,000 & 4,000,000 & 0.000000 s & 0.000000 s & 0.000000 s
\\\addlinespace
\bottomrule
\end{longtable}

Before launching the console application for a square matrix of 1,000 x
1,000, the memory is low (but Visual Studio takes a big part of memory).

At the end of the console application for a square matrix of 1,000 x
1,000, the memory is full because all pointers take a lot of space and
there are 1,000,000 of threads in memory.

When the application finished, all the memory is released.

\begin{quote}
\textbf{NOTE} Not using the `Cell' structure as in the benchmarks 3 and
4.
\end{quote}

\hyperdef{}{benchmark-3}{\subsection{Benchmark 3}\label{benchmark-3}}

The third benchmark is created by executing the parallel code with CPU
in C++ using pthread. The code uses dynamic arrays and a dynamic number
of threads. Each row in the result matrix is performed by one thread.

\begin{longtable}[c]{@{}lllll@{}}
\toprule\addlinespace
Matrix Dimension & Number of cells & Elapsed Time 1 & Elapsed Time 2 &
Elapsed Time 3
\\\addlinespace
\midrule\endhead
2 x 2 & 4 & 0.000655 s & 0.000203 s & 0.000496 s
\\\addlinespace
25 x 25 & 625 & 0.010548 s & 0.011697 s & 0.016335 s
\\\addlinespace
50 x 50 & 2,500 & 0.022197 s & 0.023205 s & 0.029736 s
\\\addlinespace
100 x 100 & 10,000 & 0.053766 s & 0.074381 s & 0.043908 s
\\\addlinespace
200 x 200 & 40,000 & 0.118335 s & 0.112993 s & 0.104711 s
\\\addlinespace
500 x 500 & 250,000 & 0.907414 s & 1.284270 s & 1.133060 s
\\\addlinespace
1,000 x 1,000 & 1,000,000 & 8.618210 s & 8.505290 s & 8.367430 s
\\\addlinespace
1,500 x 1,500 & 2,250,000 & 25.71800 s & 27.26910 s & 25.96100 s
\\\addlinespace
2,000 x 2,000 & 4,000,000 & 51.79730 s & 32.06580 s & 50.01410 s
\\\addlinespace
\bottomrule
\end{longtable}

In the figure below you can see the performance of the computer when the
mutiplication of two 2000 x 2000 matrix is performed.

In the figure below you can see that the memory empties at the end of
the program and then all memory is free when the executable is closed.

\begin{quote}
\textbf{NOTE} Using the `Cell' structure.
\end{quote}

\hyperdef{}{benchmark-4}{\subsection{Benchmark 4}\label{benchmark-4}}

The fourth benchmark is created by executing the parallel code with CPU
in C++ using pthread. The code uses dynamic arrays and a dynamic number
of threads. Cells in the result matrix are performed by one of the
threads according to a shifting algorithm.

\subsubsection{Number of threads: 2}\label{number-of-threads-2}

\begin{longtable}[c]{@{}lllll@{}}
\toprule\addlinespace
Matrix Dimension & Number of cells & Elapsed Time 1 & Elapsed Time 2 &
Elapsed Time 3
\\\addlinespace
\midrule\endhead
2 x 2 & 4 & 0.003897 s & 0.001216 s & 0.001205 s
\\\addlinespace
25 x 25 & 625 & 0.001716 s & 0.001528 s & 0.001156 s
\\\addlinespace
50 x 50 & 2,500 & 0.006530 s & 0.002301 s & 0.002024 s
\\\addlinespace
100 x 100 & 10,000 & 0.007007 s & 0.052376 s & 0.011554 s
\\\addlinespace
200 x 200 & 40,000 & 0.056970 s & 0.046925 s & 0.050223 s
\\\addlinespace
500 x 500 & 250,000 & 0.920728 s & 1.089180 s & 1.085580 s
\\\addlinespace
1,000 x 1,000 & 1,000,000 & 11.11410 s & 10.93310 s & 10.65640 s
\\\addlinespace
1,500 x 1,500 & 2,250,000 & 39.03960 s & 37.91340 s & 38.91380 s
\\\addlinespace
2,000 x 2,000 & 4,000,000 & 101.9630 s & 102.7530 s & 100.9980 s
\\\addlinespace
\bottomrule
\end{longtable}

\subsubsection{Number of threads: 3}\label{number-of-threads-3}

\begin{longtable}[c]{@{}lllll@{}}
\toprule\addlinespace
Matrix Dimension & Number of cells & Elapsed Time 1 & Elapsed Time 2 &
Elapsed Time 3
\\\addlinespace
\midrule\endhead
2 x 2 & 4 & 0.001545 s & 0.001774 s & 0.030921 s
\\\addlinespace
25 x 25 & 625 & 0.001572 s & 0.007728 s & 0.001804 s
\\\addlinespace
50 x 50 & 2,500 & 0.003861 s & 0.003202 s & 0.005808 s
\\\addlinespace
100 x 100 & 10,000 & 0.006439 s & 0.011860 s & 0.011267 s
\\\addlinespace
200 x 200 & 40,000 & 0.039320 s & 0.059699 s & 0.054823 s
\\\addlinespace
500 x 500 & 250,000 & 0.713641 s & 0.761757 s & 0.725308 s
\\\addlinespace
1,000 x 1,000 & 1,000,000 & 8.023100 s & 7.817020 s & 7.643520 s
\\\addlinespace
1,500 x 1,500 & 2,250,000 & 29.89010 s & 30.31270 s & 30.57400 s
\\\addlinespace
2,000 x 2,000 & 4,000,000 & 92.73030 s & 83.12000 s & 86.39570 s
\\\addlinespace
\bottomrule
\end{longtable}

\subsubsection{Number of threads: 4}\label{number-of-threads-4}

\begin{longtable}[c]{@{}lllll@{}}
\toprule\addlinespace
Matrix Dimension & Number of cells & Elapsed Time 1 & Elapsed Time 2 &
Elapsed Time 3
\\\addlinespace
\midrule\endhead
2 x 2 & 4 & 0.003548 s & 0.002098 s & 0.031445 s
\\\addlinespace
25 x 25 & 625 & 0.002542 s & 0.002289 s & 0.001970 s
\\\addlinespace
50 x 50 & 2,500 & 0.004001 s & 0.003347 s & 0.002551 s
\\\addlinespace
100 x 100 & 10,000 & 0.016078 s & 0.025590 s & 0.041590 s
\\\addlinespace
200 x 200 & 40,000 & 0.054145 s & 0.049041 s & 0.071671 s
\\\addlinespace
500 x 500 & 250,000 & 0.593204 s & 0.782953 s & 0.628277 s
\\\addlinespace
1,000 x 1,000 & 1,000,000 & 7.991940 s & 7.606290 s & 7.705700 s
\\\addlinespace
1,500 x 1,500 & 2,250,000 & 26.71180 s & 24.19720 s & 24.50140 s
\\\addlinespace
2,000 x 2,000 & 4,000,000 & 77.18820 s & 73.15850 s & 74.93440 s
\\\addlinespace
\bottomrule
\end{longtable}

\subsubsection{Number of threads: 8}\label{number-of-threads-8}

\begin{longtable}[c]{@{}lllll@{}}
\toprule\addlinespace
Matrix Dimension & Number of cells & Elapsed Time 1 & Elapsed Time 2 &
Elapsed Time 3
\\\addlinespace
\midrule\endhead
2 x 2 & 4 & 0.003808 s & 0.003520 s & 0.037883 s
\\\addlinespace
25 x 25 & 625 & 0.004066 s & 0.026112 s & 0.004563 s
\\\addlinespace
50 x 50 & 2,500 & 0.007793 s & 0.006199 s & 0.004467 s
\\\addlinespace
100 x 100 & 10,000 & 0.013294 s & 0.018714 s & 0.009567 s
\\\addlinespace
200 x 200 & 40,000 & 0.059391 s & 0.042749 s & 0.041908 s
\\\addlinespace
500 x 500 & 250,000 & 0.631150 s & 0.729812 s & 0.594458 s
\\\addlinespace
1,000 x 1,000 & 1,000,000 & 7.877320 s & 8.428020 s & 8.243770 s
\\\addlinespace
1,500 x 1,500 & 2,250,000 & 32.18740 s & 30.61070 s & 29.90000 s
\\\addlinespace
2,000 x 2,000 & 4,000,000 & 89.21210 s & 89.62580 s & 91.95300 s
\\\addlinespace
\bottomrule
\end{longtable}

\begin{quote}
\textbf{NOTE} Using the `Cell' structure.
\end{quote}

\hyperdef{}{benchmark-5}{\subsection{Benchmark 5}\label{benchmark-5}}

The fifth benchmark is created by executing the parallel code with GPU
in CUDA C++ using pthread. The code uses dynamic arrays and a dynamic
number of threads. Each cell in the result matrix is performed by one
thread.

\begin{longtable}[c]{@{}lllll@{}}
\toprule\addlinespace
Matrix Dimension & Number of cells & Elapsed Time 1 & Elapsed Time 2 &
Elapsed Time 3
\\\addlinespace
\midrule\endhead
2 x 2 & 4 & 0.000126 s & 0.000085 s & 0.000083 s
\\\addlinespace
25 x 25 & 625 & 0.000096 s & 0.000091 s & 0.000142 s
\\\addlinespace
50 x 50 & 2,500 & 0.000159 s & 0.000177 s & 0.000185 s
\\\addlinespace
100 x 100 & 10,000 & 0.000546 s & 0.000498 s & 0.000567 s
\\\addlinespace
200 x 200 & 40,000 & 0.004514 s & 0.004495 s & 0.004509 s
\\\addlinespace
500 x 500 & 250,000 & 0.066477 s & 0.066450 s & 0.066408 s
\\\addlinespace
1,000 x 1,000 & 1,000,000 & 0.537625 s & 0.536510 s & 0.538056 s
\\\addlinespace
1,500 x 1,500 & 2,250,000 & 1.855070 s & 1.845630 s & 1.853430 s
\\\addlinespace
2,000 x 2,000 & 4,000,000 & 2.736540 s & 3.487340 s & 3.270900 s
\\\addlinespace
\bottomrule
\end{longtable}

\hyperdef{}{conclusion}{\section{Conclusion}\label{conclusion}}

No conclusion yet

\hyperdef{}{contribute}{\section{Contribute}\label{contribute}}

No one has contribute to the project.

\hyperdef{}{bugs}{\section{Bugs}\label{bugs}}

\begin{itemize}
\itemsep1pt\parskip0pt\parsep0pt
\item
  Error with the memory deleting:

  \begin{itemize}
  \itemsep1pt\parskip0pt\parsep0pt
  \item
    detected the 01 May 2016
  \item
    solved the 01 May 2016
  \end{itemize}
\item
  Error on benchmark 4 with matrix n x n where n \textgreater{} 2

  \begin{itemize}
  \itemsep1pt\parskip0pt\parsep0pt
  \item
    detected the 30 April 2016
  \item
    solved the 19 May 2016 by using pthread\_join() function.
  \end{itemize}
\end{itemize}

\end{document}
